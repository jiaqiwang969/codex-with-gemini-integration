\documentclass[UTF8,a4paper]{ctexart}
\usepackage{geometry}
\geometry{left=2.2cm,right=2.2cm,top=2.4cm,bottom=2.4cm}
\usepackage{hyperref}
\hypersetup{colorlinks=true,linkcolor=blue,urlcolor=blue}
\usepackage{longtable}
\usepackage{array}
\usepackage{enumitem}
\setlist{nosep}

\title{Codex 自定义版本功能对标报告\\(基于官方 v0.72.0-alpha.2 的增量差异)}
\author{基于目录:\texttt{codex} vs \texttt{codex-官方-0.72-alpha2}}
\date{\today}

\begin{document}
\maketitle

\section{报告目的与范围}
本报告对比当前自定义 Codex 工程(\texttt{/Users/jqwang/136-codex-5.2-更新/codex})与官方 v0.72.0-alpha.2(\texttt{/Users/jqwang/136-codex-5.2-更新/codex-官方-0.72-alpha2})的差异,
总结自定义版本在官方基线之上额外添加的功能与改动点。

对比方式:
\begin{itemize}
  \item 文件级差异:统计 ``Only in custom'' 与 ``Files differ'' 的集合。
  \item 关键字与入口追踪:从 CLI/TUI/核心协议/模型提供方/MCP 工具等入口定位功能。
  \item 只聚焦功能与行为层差异;构建/CI 细节改动仅作附录提及。
\end{itemize}

\paragraph{维护目标(用于未来升级官方版本)}
本报告的主要用途是:当官方版本升级时,可以快速定位 ``哪些是官方已有能力''、``哪些是自定义补丁'',从而在合并时减少冲突与误删。
建议升级流程:
\begin{itemize}
  \item 先更新本报告中的 ``基线版本号'' 与官方对比目录;
  \item 按本报告列出的 ``自定义模块/补丁'' 逐项核对:是否已被官方吸收、是否需要迁移到新 API、是否需要保留;
  \item 对高风险差异(跨 provider、协议字段、历史记录格式、TUI 事件链路)优先做回归测试。
\end{itemize}

\section{总体差异概览}
自定义版本在官方 v0.72.0-alpha.2 基础上新增/引入的主要模块如下(按功能聚类)。说明:官方 v0.72 系列已内置 GPT-5.2 等模型族与相关提示词,本节仅列出自定义增量功能。

{\small
\begin{longtable}{>{\raggedright\arraybackslash}p{0.32\textwidth}>{\raggedright\arraybackslash}p{0.62\textwidth}}
\textbf{新增模块} & \textbf{对应功能}\\ \hline
\path{codex-rs/cxresume} & 独立的会话恢复/浏览工具(Node/TUI),并在 Codex TUI 内嵌会话选择器。\\
\path{codex-rs/tumix} & TUMIX 多智能体并行执行框架(Round1 15 专家并行)。\\
\path{codex-rs/multi-agent} & 通用多智能体配置/编排与 ``delegate\_agent'' 代理工具。\\
\path{codex-rs/git-graph} & vendored \texttt{git-graph} 库与命令;TUI 侧提供可视化 Git 提交图。\\
\path{codex-rs/external/appium-mcp} & Appium MCP 服务器(移动端自动化工具集)。\\
\path{codex-rs/hunyuan-mcp-server} & 腾讯云混元 AI3D MCP 服务器(文/图生 3D)。\\
\path{core/gemini_*_prompt.md} & Gemini / Germini 系列系统提示词。\\
\path{tui/src/chatwidget.rs} & Germini 文生图:保存模型生成的图片、\texttt{/open-image} 打开最近生成图、\texttt{/ref-image} 管理参考图。\\
\path{core/src/sandboxing/assessment.rs} & 模型驱动的 sandbox 风险评估。\\
\path{tui/src/*session_*} & 会话别名、底部 session bar、内嵌 cxresume 选择器等 UI 增强。\\
\end{longtable}
}

\section{自定义功能详解}

\subsection{会话管理增强:resume、clone、别名与 cxresume}
\paragraph{1) Session Clone(resume-clone)}
官方基线(v0.72.0-alpha.2)仅支持 \texttt{codex exec resume} 继续原会话。自定义版本新增 ``会话克隆'':
\begin{itemize}
  \item CLI 新增子命令:\texttt{codex exec resume-clone <session\_id> [prompt]}。
  \item Core 新增 \texttt{ConversationManager::clone\_conversation\_from\_rollout}:读取原 rollout 全历史,以 \texttt{Forked} 初始历史启动新会话,新会话拥有全新 session\_id,且不修改原文件。
  \item 该能力成为 Tumix 等多智能体并行的基础(用于复制上下文)。 
\end{itemize}

\paragraph{2) cxresume(会话恢复/浏览器)}
自定义版本引入独立工具 \texttt{cxresume}(Node.js + 分屏 TUI),用于浏览 \texttt{\~/.codex/sessions} 的历史会话并一键恢复:
\begin{itemize}
  \item 支持全量/按 cwd 过滤、搜索、预览、删除、复制 session id 等。
  \item Codex TUI 内新增 \texttt{cxresume\_picker\_widget} 作为 overlay,支持在 GUI 内直接挑选历史会话。
  \item App 侧增加预热/缓存逻辑,提升打开 picker 速度。
\end{itemize}

\paragraph{3) Session Alias(会话命名)与底部 Session Bar}
\begin{itemize}
  \item 新增 \texttt{SessionAliasManager}:在 \texttt{\~/.codex/session\_aliases.json} 维护 session\_id $\rightarrow$ 别名映射,支持设置/删除/持久化。
  \item 新增 \texttt{SessionAliasInput}:新会话创建或重命名时弹出输入框,让用户命名会话。
  \item 新增底部 \texttt{SessionBar}(类似 tmux):展示当前 cwd 相关的历史会话标签(优先别名,其次首条用户消息短摘要),并支持快速左右切换/新建。
\end{itemize}

\subsection{通用多智能体系统与 Delegate 工具}
自定义版本加入可配置的多智能体编排能力:
\begin{itemize}
  \item 新增 crate:\texttt{codex-multi-agent},负责:
    \begin{itemize}
      \item 读取 \texttt{\~/.codex/agents/<agent\_id>/} 下的子智能体配置;
      \item 维护允许调用的 agent 列表;
      \item 编排子会话的启动/事件回传/会话切换(含 Detached 后台模式)。
    \end{itemize}
  \item Core 新增 ``delegate\_agent'' 工具与 \texttt{DelegateToolAdapter} 抽象:
    \begin{itemize}
      \item 支持单次委派或 batch 委派;
      \item Invocation mode:\texttt{immediate}(阻塞返回摘要)与 \texttt{detached}(后台运行)。
    \end{itemize}
  \item TUI 新增 \texttt{/agent} slash command,用于切入某个 delegated 会话继续交互。
\end{itemize}

\subsection{TUMIX 多智能体并行执行框架}
TUMIX 是自定义版本新增的 ``一键多专家并行'' 能力:
\begin{itemize}
  \item 新增 crate:\texttt{codex-tumix},核心流程:
    \begin{enumerate}
      \item Meta-agent 生成 15 个专家 agent 配置;
      \item 为每个专家创建独立 Git worktree;
      \item 基于 \texttt{resume-clone} 克隆父会话上下文,15 个 agent 并行执行;
      \item 每个 worktree 自动提交变更并记录 session\_id;
      \item 结果写入 \texttt{.tumix/round1\_sessions.json}。
    \end{enumerate}
  \item CLI 新增:\texttt{codex tumix <parent\_session\_id>}。
  \item TUI 新增:\texttt{/tumix} 与 \texttt{/tumix-stop} slash command;会话列表中显示 Tumix 运行状态徽标。
\end{itemize}

\subsection{Git Graph 可视化}
自定义版本内嵌 \texttt{git-graph} 库与 TUI 展示:
\begin{itemize}
  \item 新增 vendored crate:\texttt{codex-rs/git-graph}(来源 \texttt{mlange-42/git-graph})。
  \item TUI 新增 Git Graph overlay(快捷键 \texttt{Ctrl+G}):
    \begin{itemize}
      \item 优先使用库生成 ``圆角 Unicode'' 提交图(含彩色分支);
      \item 失败时回退到 \texttt{git log --graph} 并做 Unicode 转换;
      \item overlay 支持滚动、刷新、全历史展示。
    \end{itemize}
\end{itemize}

\subsection{Appium MCP(移动端自动化)}
自定义版本 vendored 了 Appium MCP 服务器:
\begin{itemize}
  \item 新增目录 \texttt{codex-rs/external/appium-mcp}(TypeScript)。
  \item 通过 MCP 向 Codex 暴露移动端自动化工具:设备选择、session 管理、元素查找/点击/输入、截图、滚动、生成测试脚本等。
  \item 该模块可在用户的 MCP 配置中按需启用。
\end{itemize}

\subsection{Gemini / Germini 系列模型与图像模型}
自定义版本大幅扩展了 Gemini 家族模型的支持:
\begin{itemize}
  \item 内置 provider ``gemini'':
    \begin{itemize}
      \item 默认指向 \texttt{https://api.ppchat.vip/v1beta},可通过 \texttt{GEMINI\_BASE\_URL} 覆盖;
      \item 认证优先使用环境变量 \texttt{GEMINI\_API\_KEY}(映射到 \texttt{X-Goog-Api-Key}),否则回退复用 \texttt{OPENAI\_API\_KEY};
      \item 支持可选 \texttt{GEMINI\_COOKIE}。
    \end{itemize}
  \item 新增 model presets(在模型选择器可见):
    \begin{itemize}
      \item \texttt{gemini-3-pro-preview}:基础 Gemini 3 Pro。
      \item \texttt{gemini-3-pro-preview-codex}:使用 Codex 风格 system prompt。
      \item \texttt{gemini-3-pro-preview-thinking} / \texttt{-thinking-codex}:thinking 变体。
      \item \texttt{gemini-3-pro-preview-thinking-germini}:Germini(Gemini CLI)风格系统提示词。
      \item \texttt{gemini-3-pro-image-preview}:图像理解/生成模型。
    \end{itemize}
  \item Prompt 体系:新增 \texttt{gemini\_prompt.md}、\texttt{gemini\_codex\_prompt.md}、\texttt{gemini\_germini\_prompt.md},分别对应基础/Codex 优化/Germini 工作流。
  \item 自动 provider 切换:当用户在 TUI/CLI 选择 \texttt{gemini-*} 模型时自动切换到 ``gemini'' provider;反之从 Gemini 切回 GPT 模型时自动回到 ``openai/openai-proxy''。
  \item 协议增强:
    \begin{itemize}
      \item 支持 Gemini 3 的 \texttt{thought\_signature} 透传与回填(避免工具调用校验失败)。
      \item 对非 Gemini provider 自动剥离 \texttt{thought\_signature},保持兼容。
    \end{itemize}
\end{itemize}

\paragraph{Germini 文生图(Gemini 3 Pro Image)}
除 ``Germini 风格系统提示词'' 外,自定义版本还补齐了文生图/图像迭代的 TUI 交互链路(详见 \texttt{codex-rs/docs/gemini\_3\_pro\_image\_user\_guide.tex})。核心能力(凝练版):
\begin{itemize}
  \item \textbf{模型输出落盘}:当 \texttt{gemini-3-pro-image-preview} 在 assistant message 中返回 \texttt{input\_image}(data URL),TUI 自动保存到 \texttt{\~/.codex/images/<conversation\_id>/000000.<ext>},并提示 \texttt{/open-image} 打开最近生成的图片(实现点:\texttt{tui/src/chatwidget.rs})。
  \item \textbf{参考图 RefSet 管理}:\texttt{/ref-image <paths...> [-- prompt]} 把本地图片转换为 data URL 并持久化为 ``本会话参考图集合'';\texttt{/ref-image clear} 清空;\texttt{/ref-image ls} 查看(实现点:\texttt{tui/src/chatwidget.rs} + core 的 \texttt{Op::SetReferenceImages} 处理)。
  \item \textbf{隐式继续改图}:若未显式设置参考图,core 会优先使用本轮用户附图;若本轮无图则可回退到历史中最近一张 assistant 图片作为参考图,从而支持 ``生成图 $\rightarrow$ 继续改图'' 的自然循环(实现点:\texttt{core/src/codex.rs} 的 reference\_images 推导逻辑)。
\end{itemize}

\paragraph{配置与 Provider 自由切换(Gemini/Germini $\leftrightarrow$ GPT)}
该自定义版本对 ``模型(model)'' 与 ``提供方(provider / wire\_api)'' 的关系做了 ``可自由切换'' 的工程化处理,目的是在同一会话内让用户可以在 Gemini 家族与 GPT 家族间来回切换,而不需要手动修改大量配置。
\begin{itemize}
  \item \textbf{核心配置字段}:\texttt{model} 决定模型 slug;配置层用 \texttt{model\_provider}(内部解析为 \texttt{model\_provider\_id})决定请求发往哪个 provider;provider 通过 \texttt{wire\_api} 选择协议(OpenAI Responses/Chat vs Gemini)。
  \item \textbf{自动切换策略(保守)}:仅在 ``gemini'' 与 ``openai/openai-proxy'' 间自动切换;第三方 provider(如 openrouter)不会被自动切换,必须显式选择以避免意外路由。
  \item \textbf{切换入口(TUI)}:模型选择器会发送 \texttt{Op::OverrideTurnContext\{model=...\}} 到 core,core 在 \texttt{SessionConfiguration::apply} 内结合 \texttt{Config::preferred\_model\_provider\_id\_for\_model} 判断是否需要同步切 provider。
  \item \textbf{切换入口(CLI)}:建议用 profile 固定 provider,再通过 \texttt{-m <model>}(或配置文件)切换 model;也可切换 profile 达到 ``模型 + provider'' 组合切换。
\end{itemize}

\paragraph{配置示例(\texttt{\~/.codex/config.toml})}
下面示例展示 ``Gemini(含 Germini)'' 与 ``OpenAI 风格(GPT)'' 两个 provider 并存,并支持在 TUI 内直接切换模型时自动切换 provider(示例仅展示关键字段):
\begin{verbatim}
[model_providers.gemini]
name = "gemini"
wire_api = "gemini"
base_url = "https://api.ppchat.vip/v1beta"
env_key = "GEMINI_API_KEY"

[model_providers.openai-proxy]
name = "openai-proxy"
wire_api = "responses"
base_url = "https://api.openai.com/v1"
env_key = "OPENAI_API_KEY"

[profiles.gemini]
model = "gemini-3-pro-preview-thinking-germini"
model_provider = "gemini"

[profiles.gpt]
model = "gpt-4.1"
model_provider = "openai-proxy"
\end{verbatim}

\paragraph{兼容性修复:Gemini 图像会话切换到 GPT 触发 429/400}
该问题是 ``跨 provider + 图像历史'' 的典型升级冲突点,官方更新时最容易被覆盖或改坏,建议在升级前优先核对该段逻辑是否仍存在。
\begin{itemize}
  \item \textbf{现象(用户可见)}:先用 \texttt{gemini-3-pro-image-preview} 进行图像对话,再切换到 GPT 模型后发送消息,出现:
    \begin{itemize}
      \item 429(proxy 或上游限流/体积限制),并最终报 ``exceeded retry limit'';
      \item 随后暴露 400:\texttt{Invalid value: 'input\_text'. Supported values are 'output\_text' and 'refusal'.}(\texttt{param: input[i].content[j]})。
    \end{itemize}
  \item \textbf{根因拆解}:
    \begin{itemize}
      \item Gemini 图像模型会把图片以 \texttt{data:image/...;base64,...} 形式写入历史(\texttt{input\_image})。切换到 GPT provider 后,这段 ``超大图片历史'' 仍会进入下一次 OpenAI-style 请求,容易触发 proxy 的限流/体积策略(表现为 429)。
      \item 初版 ``清理图片'' 逻辑把所有图片都替换成 \texttt{input\_text};但 OpenAI Responses API 对 \texttt{role=assistant} 的 message 内容只接受 \texttt{output\_text/refusal},因此出现 400 invalid\_value。
    \end{itemize}
  \item \textbf{修复策略(自定义补丁)}:
    \begin{itemize}
      \item \textbf{模型切换时清理}:当 model 从 \texttt{gemini-*} 切到非 \texttt{gemini-*} 时,统一将历史中的 \texttt{input\_image} 替换为占位文本,避免把大图带入 GPT 请求(核心点在 \texttt{core/src/codex.rs} 与 \texttt{core/src/context\_manager/history.rs})。
      \item \textbf{按角色替换(关键)}:\texttt{role=user} 的图片替换为 \texttt{input\_text};\texttt{role=assistant} 的图片替换为 \texttt{output\_text},以满足 Responses API 的 schema(对应 400 修复)。
      \item \textbf{429 诊断与等待}:429 时解析并展示上游 \texttt{error.message},并优先按 \texttt{Retry-After} 或 ``try again in Xs'' 等建议等待时长重试,避免在限流窗口内耗尽重试次数(\texttt{core/src/api\_bridge.rs}、\texttt{core/src/error.rs}、\texttt{core/src/codex.rs}、\texttt{core/src/util.rs})。
    \end{itemize}
  \item \textbf{定位线索(便于未来合并)}:本补丁集中在 commit \texttt{69d25370}(message:\texttt{core: clear Gemini images on model switch}),相关单测覆盖位于 \texttt{core/src/codex.rs} 与 \texttt{core/src/context\_manager/history\_tests.rs}。
\end{itemize}

\subsection{腾讯云混元 AI3D MCP 集成}
自定义版本新增 ``混元 AI3D'' 能力:
\begin{itemize}
  \item 新增 crate:\texttt{codex-hunyuan-mcp-server}。
  \item 通过 MCP 暴露:
    \begin{itemize}
      \item 文生 3D、图生 3D、Sketch(文+图)模式;
      \item 支持 Professional/Rapid/Standard 三种 API;
      \item 自动提交任务、轮询状态并下载结果。
    \end{itemize}
  \item 输出文件默认落盘到 \texttt{/tmp/hunyuan-3d/} 按时间与 JobID 归档。
\end{itemize}

\subsection{Sandbox 风险评估与审批 UI}
除官方已有 sandbox/审批外,自定义版本增加了 ``模型评估风险'' 的实验特性:
\begin{itemize}
  \item Feature flag:\texttt{experimental\_sandbox\_command\_assessment}。
  \item 当 sandbox 阻止命令时,core 调用模型生成 \texttt{SandboxCommandAssessment}:
    \begin{itemize}
      \item 返回风险等级(low/medium/high)与摘要;
      \item 评估 prompt 由 \texttt{core/templates/sandboxing/assessment\_prompt.md} 渲染。
    \end{itemize}
  \item 协议与 TUI:
    \begin{itemize}
      \item 审批事件携带 \texttt{risk} 字段;
      \item Approval overlay 顶部显示风险摘要与等级。
    \end{itemize}
  \item exec 协议字段调整:将 \texttt{sandbox\_permissions} 替换为 \texttt{with\_escalated\_permissions},便于与新的批准策略一致。
\end{itemize}

\section{结论}
相较官方 v0.72.0-alpha.2,自定义版本形成了 ``多模型(Gemini/Germini)+ 多智能体(delegate/TUMIX)+ 会话工程化(clone/cxresume/别名)+ MCP 工具扩展(Appium/Hunyuan AI3D)+ 开发体验增强(Git Graph/风险评估)+ 图像工作流(Germini 文生图)'' 的完整扩展体系。

这些增强使 Codex 从单一代码助手升级为可在复杂工程中进行并行分工、跨模态生成与强会话管理的终端/GUI 生产力平台。

\end{document}
