\documentclass[12pt,a4paper]{article}
\usepackage{xeCJK}
\usepackage{geometry}
\usepackage{graphicx}
\usepackage{xcolor}
\usepackage{listings}
\usepackage{tcolorbox}
\usepackage{enumitem}
\usepackage{tikz}
\usepackage{booktabs}
\usepackage{hyperref}
\usepackage{fancyhdr}
\usetikzlibrary{shapes.geometric, arrows, positioning, calc}

\geometry{margin=2.5cm}
\setCJKmainfont{PingFang SC}
\setmainfont{Helvetica Neue}

\definecolor{codebg}{RGB}{245,245,245}
\definecolor{warnbg}{RGB}{255,243,224}
\definecolor{dangerbg}{RGB}{255,235,238}
\definecolor{infobg}{RGB}{227,242,253}
\definecolor{successbg}{RGB}{232,245,233}

\tcbuselibrary{skins,breakable}

\newtcolorbox{warningbox}{
  colback=warnbg,
  colframe=orange!70!black,
  title={\textbf{警告}},
  fonttitle=\bfseries,
  breakable
}

\newtcolorbox{dangerbox}{
  colback=dangerbg,
  colframe=red!70!black,
  title={\textbf{严重问题}},
  fonttitle=\bfseries,
  breakable
}

\newtcolorbox{infobox}{
  colback=infobg,
  colframe=blue!70!black,
  title={\textbf{说明}},
  fonttitle=\bfseries,
  breakable
}

\newtcolorbox{successbox}{
  colback=successbg,
  colframe=green!60!black,
  title={\textbf{解决方案}},
  fonttitle=\bfseries,
  breakable
}

\newtcolorbox{keybox}{
  colback=red!5,
  colframe=red!80!black,
  title={\textbf{关键发现}},
  fonttitle=\bfseries,
  breakable
}

\newtcolorbox{conclusionbox}{
  colback=blue!5,
  colframe=blue!80!black,
  title={\textbf{结论}},
  fonttitle=\bfseries,
  breakable
}

\hypersetup{
    colorlinks=true,
    linkcolor=blue,
    urlcolor=blue,
    citecolor=blue
}

\pagestyle{fancy}
\fancyhf{}
\fancyhead[L]{Codex Compact 回滚问题分析}
\fancyhead[R]{\thepage}
\fancyfoot[C]{基于源码分析与官方 GitHub Issues 调研}

\title{\textbf{Codex CLI Compact 机制\\导致代码自动回滚问题分析报告}}
\author{技术问题报告 v4.0\\[1em]\small 基于 codex-rs 源码分析、官方 GitHub Issues 调研与第一性原理分析}
\date{\today}

\begin{document}

\maketitle

\begin{abstract}
本报告分析了 Codex CLI 在 tmux 多窗口环境下,由于 compact(上下文压缩)机制触发后,agent 自动执行 \texttt{git diff} 并主动回滚代码的问题。经过源码分析和官方 GitHub Issues 调研,确认\textbf{这是一个已知的普遍问题}。本报告进一步通过\textbf{第一性原理分析},解释了为什么官方提供的 execpolicy 方案\textbf{无法根本解决问题},并指出真正需要修复的是 Compact 机制本身。
\end{abstract}

\tableofcontents
\newpage

%==============================================================================
\section{问题概述}
%==============================================================================

\subsection{问题现象}

\begin{keybox}
\textbf{核心现象}:Codex compact 后,agent 会自动执行 \texttt{git diff},然后\textbf{主动回滚}代码到之前的状态,导致用户的工作丢失。

\vspace{0.5em}
\textbf{用户反馈}:
\begin{itemize}
    \item "codex compact后开始git diff然后回滚这个问题 看来依然没有修复"
    \item "我的自动git push及时避免了一次灾难回滚的发生"
    \item "临时解决不回滚就行了,先不耽误干活就成"
\end{itemize}
\end{keybox}

\subsection{问题触发条件}

\begin{warningbox}
该问题\textbf{主要发生在 tmux 多窗口环境}下:
\begin{itemize}
    \item 多个 agent 同时工作,加速 token 消耗
    \item 更容易触发 auto-compact
    \item 多 agent 共享同一个 git 仓库,状态更复杂
\end{itemize}
\end{warningbox}

\subsection{问题影响}

\begin{enumerate}
    \item \textbf{代码丢失}:Agent 之前的修改被撤销
    \item \textbf{重复工作}:Agent 忘记之前做了什么,重复执行相同操作
    \item \textbf{信任危机}:用户不敢让 agent 长时间独立工作
\end{enumerate}

%==============================================================================
\section{问题发生流程}
%==============================================================================

\subsection{完整流程图}

\begin{center}
\begin{tikzpicture}[
    node distance=0.6cm,
    box/.style={rectangle, draw, rounded corners, minimum width=3cm, minimum height=0.8cm, align=center, font=\small},
    arrow/.style={->, thick, >=stealth},
]

% Row 1 - Normal work
\node[box, fill=green!20] (t1) {T1: 开始工作\\创建 Ghost Snapshot};
\node[box, fill=blue!20, right=0.4cm of t1] (t2) {T2: 修改代码\\写了很多文件};
\node[box, fill=blue!20, right=0.4cm of t2] (t3) {T3: 继续工作\\可能已 commit};
\node[box, fill=orange!40, right=0.4cm of t3] (t4) {T4: Token 超限\\触发 Compact};

% Row 2 - After compact
\node[box, fill=yellow!30, below=1.2cm of t1] (t5) {T5: Agent "失忆"\\只有模糊摘要};
\node[box, fill=yellow!30, right=0.4cm of t5] (t6) {T6: 执行 git diff\\查看当前状态};
\node[box, fill=red!30, right=0.4cm of t6] (t7) {T7: 发现"异常"\\有未知的修改};
\node[box, fill=red!50, right=0.4cm of t7] (t8) {T8: 主动回滚\\git checkout .};

% Row 3 - Result
\node[box, fill=gray!40, below=1.2cm of t6, minimum width=6cm] (t9) {T9: 代码丢失,工作白费};

% Arrows
\draw[arrow] (t1) -- (t2);
\draw[arrow] (t2) -- (t3);
\draw[arrow] (t3) -- (t4);
\draw[arrow] (t4.south) -- ++(0,-0.4) -| (t5.north);
\draw[arrow] (t5) -- (t6);
\draw[arrow] (t6) -- (t7);
\draw[arrow] (t7) -- (t8);
\draw[arrow] (t8.south) -- ++(0,-0.4) -| (t9.east);

% Labels
\node[above=0.1cm of t1, font=\footnotesize\bfseries, color=green!50!black] {正常工作阶段};
\node[above=0.1cm of t5, font=\footnotesize\bfseries, color=red!50!black] {Compact 后的错误行为};

\end{tikzpicture}
\end{center}

\subsection{Agent 的错误"思考过程"}

\begin{infobox}
Compact 后 agent 的"推理":

\begin{enumerate}
    \item 我只记得一个模糊的摘要:"正在进行代码开发..."
    \item 让我执行 \texttt{git diff} 看看当前状态
    \item 发现有很多文件被修改了
    \item \textbf{但我不记得这些是我做的}
    \item 这些修改可能是:
    \begin{itemize}
        \item 遗留的脏数据?
        \item 其他人/进程的修改?
        \item 需要清理的异常状态?
    \end{itemize}
    \item 为了保持工作区干净,我来执行 \texttt{git checkout .}
    \item \textbf{结果:所有修改被撤销!}
\end{enumerate}
\end{infobox}

%==============================================================================
\section{根本原因分析}
%==============================================================================

\subsection{问题链条}

\begin{center}
\begin{tikzpicture}[
    node distance=0.8cm,
    cause/.style={rectangle, draw, rounded corners, fill=red!10, minimum width=10cm, minimum height=0.9cm, align=center, font=\small},
    arrow/.style={->, thick, >=stealth},
]

\node[cause] (c1) {\textbf{根因1}:Compact 机制只保留文字摘要,丢失关键上下文};
\node[cause, below=0.5cm of c1] (c2) {\textbf{根因2}:摘要中没有 git 状态、修改文件列表、操作历史};
\node[cause, below=0.5cm of c2] (c3) {\textbf{根因3}:AGENTS.md 在 compact 后不会被重新读取};
\node[cause, below=0.5cm of c3] (c4) {\textbf{根因4}:Agent 对当前状态产生误判};
\node[cause, below=0.5cm of c4, fill=red!30] (c5) {\textbf{后果}:Agent 执行破坏性 git 命令(checkout/reset/restore)};

\draw[arrow] (c1) -- (c2);
\draw[arrow] (c2) -- (c3);
\draw[arrow] (c3) -- (c4);
\draw[arrow] (c4) -- (c5);

\end{tikzpicture}
\end{center}

\subsection{技术细节}

\subsubsection{Compact 摘要内容不足}

Compact 后,agent 的对话历史被压缩成类似这样的内容:

\begin{lstlisting}[backgroundcolor=\color{codebg},basicstyle=\ttfamily\small]
Another language model started to solve this problem
and produced a summary of its thinking process.
Use this to build on the work that has already been done...

[摘要内容]
"正在进行代码开发任务,修改了一些文件..."
\end{lstlisting}

\textbf{缺失的关键信息}:
\begin{itemize}
    \item 当前 git 分支名
    \item 最近的 commit hash
    \item 已修改的文件清单
    \item 明确说明"这些修改是之前工作的结果"
\end{itemize}

\subsubsection{AGENTS.md 在 Compact 后被忽略}

根据 GitHub Issue \#2927 和 \#5772:
\begin{itemize}
    \item Compact 后系统\textbf{不会重新读取} AGENTS.md
    \item 即使在 AGENTS.md 中写了"禁止执行 git checkout"
    \item Compact 后这些指令就失效了
\end{itemize}

%==============================================================================
\section{官方 GitHub Issues 调研}
%==============================================================================

经过搜索 OpenAI 官方 Codex CLI 的 GitHub issues,确认\textbf{这是一个已知的普遍问题}。

\subsection{高度相关的 Issues}

\begin{center}
\begin{tabular}{p{1.2cm}p{4.5cm}p{7.5cm}}
\toprule
\textbf{Issue} & \textbf{标题} & \textbf{问题描述} \\
\midrule
\href{https://github.com/openai/codex/issues/5957}{\#5957} & Auto compaction causes model to lose the plot & Agent 工作1小时,做了23个文件编辑,compact 后\textbf{否认做过任何修改} \\
\midrule
\href{https://github.com/openai/codex/issues/1736}{\#1736} & Codex sometimes reverting changes & Agent 执行 \texttt{git checkout}、\texttt{git reset} 回滚代码 \\
\midrule
\href{https://github.com/openai/codex/issues/6022}{\#6022} & Agent ignores AGENTS.md & 尽管明确禁止,agent 仍\textbf{重复执行}破坏性命令 \\
\midrule
\href{https://github.com/openai/codex/issues/4969}{\#4969} & Codex deleting uncommitted files & Agent 删除了 6GB 未提交文件 \\
\midrule
\href{https://github.com/openai/codex/issues/2927}{\#2927} & AGENTS.md ignored after compact & Compact 后 AGENTS.md 不会被重新读取 \\
\midrule
\href{https://github.com/openai/codex/issues/4106}{\#4106} & Control over auto-compaction & 用户要求能完全关闭 auto-compact \\
\bottomrule
\end{tabular}
\end{center}

\subsection{Issue \#5957 详情}

\begin{dangerbox}
\textbf{用户报告}:

Agent 使用 GPT-5-Codex 模型工作了\textbf{超过一个小时},对 6 个文件进行了 \textbf{23 次编辑}。然后 auto compaction 触发。

\textbf{Compact 后 agent 的回答}:
\begin{itemize}
    \item "我什么都没做,只是在等待批准"
    \item "那些编辑在指令之前就已经存在"
    \item "是之前的 assistant session 做的"
\end{itemize}

\textbf{用户评论}:模型并非恶意,它是\textbf{真的失去了上下文意识}。
\end{dangerbox}

\subsection{Issue \#1736 详情}

用户报告 agent 会执行以下破坏性命令:

\begin{lstlisting}[backgroundcolor=\color{codebg},basicstyle=\ttfamily\small]
git checkout -- file1.php
git checkout -- file2.php
git status
rm untracked_file.txt
\end{lstlisting}

一位用户评论:

\begin{quote}
\textit{"Getting to a point where I'm frantically committing every few lines of codes."}

(我现在每写几行代码就疯狂地 commit 一次)
\end{quote}

\textbf{这与我们用户的"定时自动 commit \& push"方案完全一致!}

%==============================================================================
\section{官方的"解决方案"评估}
%==============================================================================

\subsection{方案对比}

\begin{center}
\begin{tabular}{p{3cm}p{4cm}p{2cm}p{4cm}}
\toprule
\textbf{方案} & \textbf{描述} & \textbf{效果} & \textbf{问题} \\
\midrule
AGENTS.md 指令 & 在 AGENTS.md 中写禁止回滚的指令 & \textcolor{red}{❌ 不可靠} & Compact 后会被忽略;Agent 经常无视 \\
\midrule
Execution Policy & 用 .rules 文件禁止特定命令 & \textcolor{orange}{⚠️ 部分有效} & 只能阻止命令,不能修复误判逻辑 \\
\midrule
关闭 auto-compact & 完全禁用自动压缩 & \textcolor{orange}{⚠️ 有副作用} & 会导致 token 超限错误 \\
\midrule
修复 compact 逻辑 & 在摘要中保留关键信息 & \textcolor{gray}{❓ 未实现} & 官方没有从根本上修复 \\
\bottomrule
\end{tabular}
\end{center}

\subsection{Execution Policy 配置示例}

OpenAI 在 2025 年底添加了 execpolicy 功能:

\begin{lstlisting}[backgroundcolor=\color{codebg},basicstyle=\ttfamily\small]
# ~/.codex/rules/no-rollback.rules

# 禁止 git checkout
prefix_rule(
    pattern = ["git", "checkout"],
    decision = "forbidden",
)

# 禁止 git reset
prefix_rule(
    pattern = ["git", "reset"],
    decision = "forbidden",
)

# 禁止 git restore
prefix_rule(
    pattern = ["git", "restore"],
    decision = "forbidden",
)
\end{lstlisting}

\begin{warningbox}
\textbf{局限性}:execpolicy 只能阻止命令执行,但不能解决 agent 的误判问题。Agent 仍然会"想要"回滚,只是被阻止了。这可能导致 agent 尝试其他方式来"清理"状态。
\end{warningbox}

\subsection{官方实际做了什么}

\begin{enumerate}
    \item \textbf{添加了 execpolicy 功能}(2025年底)
    \begin{itemize}
        \item 可以配置命令黑名单
        \item 但这是"堵漏洞"而不是"修根本"
    \end{itemize}

    \item \textbf{关闭了很多 Issues 但没真正解决}
    \begin{itemize}
        \item Issue \#2927 被标记为 "stale" 后关闭
        \item Issue \#1736 被关闭但用户继续报告问题存在
        \item Issue \#6022 只是说"反馈已转交给模型训练团队"
    \end{itemize}

    \item \textbf{没有从根本上改进 compact 机制}
    \begin{itemize}
        \item 没有在 compact 摘要中加入 git 状态
        \item 没有在 compact 后重新读取 AGENTS.md
        \item 没有在 compact 后创建新的 snapshot
    \end{itemize}
\end{enumerate}

%==============================================================================
\section{用户的临时解决方案}
%==============================================================================

\subsection{定时自动 Commit \& Push}

用户设置了定时任务,每隔几分钟自动执行:

\begin{lstlisting}[backgroundcolor=\color{codebg},basicstyle=\ttfamily\small]
# 每隔几分钟自动执行
git add .
git commit -m "Auto-commit @ $(date)"
git push origin main
\end{lstlisting}

\begin{successbox}
\textbf{优点}:
\begin{itemize}
    \item 简单有效,能保住大部分代码
    \item 不需要修改 Codex 源码
    \item 即使本地被回滚,远程还有备份
\end{itemize}

\textbf{缺点}:
\begin{itemize}
    \item 产生大量碎片 commit
    \item 本地代码仍可能丢失,需要手动 pull
    \item 不是根本解决方案,"还有漏洞"
\end{itemize}
\end{successbox}

%==============================================================================
\section{第一性原理分析:为什么 Execpolicy 不起作用}
%==============================================================================

用户反馈:\textbf{"policy和rule那个我试了,不起作用"}。本节从第一性原理分析为什么 execpolicy 无法解决这个问题。

\subsection{Execpolicy 的工作原理}

根据源码分析(\texttt{core/src/exec\_policy.rs}),execpolicy 的工作机制如下:

\begin{lstlisting}[backgroundcolor=\color{codebg},basicstyle=\ttfamily\small]
// exec_policy.rs 核心逻辑
pub(crate) async fn create_exec_approval_requirement_for_command(
    exec_policy: &Arc<RwLock<Policy>>,
    ...
    command: &[String],  // <-- 只检查 shell 命令
    ...
)
\end{lstlisting}

\begin{keybox}
\textbf{关键发现}:Execpolicy \textbf{只能检查和阻止 shell 命令},它作用于 \texttt{shell} 和 \texttt{shell\_command} 工具调用。
\end{keybox}

\subsection{Agent 的多种"回滚"手段}

问题的核心不是 agent 执行了 \texttt{git checkout},而是 \textbf{agent 认为当前状态是"异常"需要"修复"}。

Agent 可以通过多种方式实现"修复":

\begin{center}
\begin{tabular}{p{4.5cm}p{3cm}p{4.5cm}}
\toprule
\textbf{操作方式} & \textbf{被 Execpolicy 检查?} & \textbf{能造成回滚?} \\
\midrule
\texttt{git checkout .} & \textcolor{green!60!black}{是} & \textcolor{red}{是} \\
\texttt{git reset --hard} & \textcolor{green!60!black}{是} & \textcolor{red}{是} \\
\texttt{git restore} & \textcolor{green!60!black}{是} & \textcolor{red}{是} \\
\midrule
Edit 工具直接覆写文件 & \textcolor{red}{否} & \textcolor{red}{是} \\
Write 工具重写文件 & \textcolor{red}{否} & \textcolor{red}{是} \\
\texttt{rm} 删除文件后重建 & \textcolor{red}{否} & \textcolor{red}{是} \\
手动读取"原始内容"后覆盖 & \textcolor{red}{否} & \textcolor{red}{是} \\
\bottomrule
\end{tabular}
\end{center}

\begin{dangerbox}
\textbf{Execpolicy 的根本局限}:

\begin{enumerate}
    \item 只能阻止 \textbf{shell 命令},无法阻止 Edit、Write、Read 等其他工具
    \item Agent 如果被禁止用 \texttt{git checkout},可能转而用 \textbf{Edit 工具} 逐个"修复"文件
    \item 即使禁止了所有 git 回滚命令,agent 仍然可以用 \textbf{Write 工具重写文件}
    \item \textbf{治标不治本}:只是堵住了部分"回滚手段",没有解决"误判"问题
\end{enumerate}
\end{dangerbox}

\subsection{问题链条的完整分析}

\begin{center}
\begin{tikzpicture}[
    node distance=0.5cm,
    box/.style={rectangle, draw, rounded corners, minimum width=12cm, minimum height=0.8cm, align=center, font=\small},
    arrow/.style={->, thick, >=stealth},
]

\node[box, fill=orange!20] (p1) {\textbf{触发}:Token 超限,触发 Compact};
\node[box, fill=yellow!20, below=0.4cm of p1] (p2) {\textbf{结果}:对话历史被压缩成简短的文字摘要};
\node[box, fill=yellow!20, below=0.4cm of p2] (p3) {\textbf{缺失}:摘要中没有 git 状态、修改文件列表、"这是我做的"声明};
\node[box, fill=red!15, below=0.4cm of p3] (p4) {\textbf{误判}:Agent 看到工作区有修改,但不知道是自己之前做的};
\node[box, fill=red!30, below=0.4cm of p4] (p5) {\textbf{行为}:Agent 尝试"清理"它认为的"异常状态"};
\node[box, fill=gray!30, below=0.4cm of p5] (p6) {\textbf{后果}:代码被回滚/覆写,工作丢失};

\draw[arrow] (p1) -- (p2);
\draw[arrow] (p2) -- (p3);
\draw[arrow] (p3) -- (p4);
\draw[arrow] (p4) -- (p5);
\draw[arrow] (p5) -- (p6);

% 标注 execpolicy 能阻止的位置
\node[right=0.3cm of p5, text=green!50!black, font=\footnotesize] {← Execpolicy 只能在这里阻止部分操作};
\node[right=0.3cm of p4, text=red!70!black, font=\footnotesize] {← 真正的问题在这里!};

\end{tikzpicture}
\end{center}

\subsection{Compact Prompt 的源码分析}

根据源码(\texttt{core/templates/compact/prompt.md}),Compact 时使用的 prompt 内容:

\begin{lstlisting}[backgroundcolor=\color{codebg},basicstyle=\ttfamily\small]
You are performing a CONTEXT CHECKPOINT COMPACTION.
Create a handoff summary for another LLM that will resume the task.

Include:
- Current progress and key decisions made
- Important context, constraints, or user preferences
- What remains to be done (clear next steps)
- Any critical data, examples, or references needed to continue

Be concise, structured, and focused on helping the next LLM
seamlessly continue the work.
\end{lstlisting}

\begin{warningbox}
\textbf{问题}:这个 prompt \textbf{没有要求} agent 记录:
\begin{enumerate}
    \item 当前 git 工作区状态
    \item 已修改的文件列表
    \item "这些修改是我(之前的 agent)做的"的明确声明
    \item 当前 git 分支和最近 commit hash
\end{enumerate}

这就是为什么 Compact 后的 agent 会"失忆"——它根本没有被要求记住这些关键信息!
\end{warningbox}

\subsection{问题发生的完整机制图}

\begin{center}
\begin{tikzpicture}[
    node distance=0.4cm,
    phase/.style={rectangle, draw, rounded corners, minimum width=14cm, minimum height=2.2cm, align=left, font=\small},
    arrow/.style={->, thick, >=stealth},
]

% Phase 1
\node[phase, fill=green!10] (ph1) {
\textbf{阶段 1:正常工作}\\[0.3em]
Agent A 工作中:\\
\hspace{1em}• 修改了 file1.rs, file2.rs, file3.rs\\
\hspace{1em}• 执行了 git diff 看过状态\\
\hspace{1em}• \textbf{知道}这些修改是自己做的
};

% Phase 2
\node[phase, fill=yellow!15, below=0.6cm of ph1] (ph2) {
\textbf{阶段 2:Compact(Token 超限)}\\[0.3em]
生成 Summary:"正在进行代码开发任务,已完成部分功能实现..."\\[0.3em]
\textcolor{red}{\textbf{没有记录}}:哪些文件被修改了、这些修改是之前 agent 做的、当前 git 状态
};

% Phase 3
\node[phase, fill=red!10, below=0.6cm of ph2] (ph3) {
\textbf{阶段 3:Compact 后的 Agent B(新的上下文)}\\[0.3em]
Agent B 看到的上下文:\\
\hspace{1em}1. Summary prefix: "Another language model started to solve..."\\
\hspace{1em}2. Summary: "正在进行代码开发任务..."(没有文件列表)\\
\hspace{1em}3. Ghost snapshots(只有 commit ID,没有文件列表)
};

% Phase 4
\node[phase, fill=red!25, below=0.6cm of ph3] (ph4) {
\textbf{阶段 4:Agent B 执行 git status / git diff}\\[0.3em]
→ 发现有很多文件被修改了\\
→ 但 summary 里没说这些是之前做的\\
→ Agent B 的推理:"这些可能是遗留的脏数据?需要清理?"\\
→ \textbf{执行"修复"操作}(git checkout 或 Edit 覆写)
};

\draw[arrow] (ph1) -- node[right, font=\footnotesize] {Token 超限} (ph2);
\draw[arrow] (ph2) -- node[right, font=\footnotesize] {上下文丢失} (ph3);
\draw[arrow] (ph3) -- node[right, font=\footnotesize] {误判} (ph4);

\end{tikzpicture}
\end{center}

\subsection{为什么用户的临时方案有效}

用户的"定时 git add \&\& git commit \&\& git push"方案之所以有效,是因为:

\begin{enumerate}
    \item 即使 agent 执行 \texttt{git checkout},只会回到\textbf{最近的 commit}
    \item 最近的 commit 就是\textbf{几分钟前}自动保存的版本
    \item 损失的只是\textbf{最近几分钟}的工作,而不是全部
    \item 远程有\textbf{完整备份},即使本地被破坏也能恢复
\end{enumerate}

\begin{infobox}
这个方案本质上是\textbf{缩短了"回滚窗口"},让回滚造成的损失可控。但它\textbf{没有解决根本问题}——agent 仍然会尝试回滚,只是回滚后的损失变小了。
\end{infobox}

\subsection{第一性原理结论}

\begin{conclusionbox}
\textbf{问题的本质}:

不是"agent 执行了 git checkout",而是\textbf{"agent 误判了当前状态"}。

\vspace{0.5em}
\textbf{Execpolicy 为什么不够}:

\begin{center}
\begin{tabular}{p{4cm}p{4cm}p{3cm}}
\toprule
\textbf{问题层面} & \textbf{具体问题} & \textbf{Execpolicy 能解决?} \\
\midrule
根本原因 & Compact summary 没有 git 状态信息 & \textcolor{red}{否} \\
直接原因 & Agent 误判当前修改是"异常" & \textcolor{red}{否} \\
表象(部分) & Agent 执行 git checkout & \textcolor{green!60!black}{部分} \\
表象(部分) & Agent 用 Edit/Write 改文件 & \textcolor{red}{否} \\
\bottomrule
\end{tabular}
\end{center}

\vspace{0.5em}
\textbf{真正需要修复的}:

\textbf{Compact 机制本身}——确保 compact 后的 agent 知道:
\begin{enumerate}
    \item 当前工作区的修改是什么
    \item 这些修改是之前工作的成果
    \item \textbf{不应该"清理"或"恢复"这些修改}
\end{enumerate}
\end{conclusionbox}

%==============================================================================
\section{建议的根本修复方向}
%==============================================================================

\begin{center}
\begin{tikzpicture}[
    node distance=0.6cm,
    fix/.style={rectangle, draw, rounded corners, fill=green!15, minimum width=12cm, minimum height=1cm, align=left, font=\small},
]

\node[fix] (f1) {\textbf{修复1}:Compact 摘要内容增强\\
\hspace{1em}• 记录当前 git branch 和最近 commit hash\\
\hspace{1em}• 列出已修改的文件清单\\
\hspace{1em}• 明确说明"这些修改是之前工作的结果,不要撤销"};

\node[fix, below=0.4cm of f1] (f2) {\textbf{修复2}:Compact 后重新加载 AGENTS.md\\
\hspace{1em}• 确保禁止指令始终生效\\
\hspace{1em}• 或将关键指令写入 compact 摘要};

\node[fix, below=0.4cm of f2] (f3) {\textbf{修复3}:Compact 后创建新的 Ghost Snapshot\\
\hspace{1em}• 保护当前状态\\
\hspace{1em}• 即使 agent 误操作,也能通过 /undo 恢复};

\node[fix, below=0.4cm of f3] (f4) {\textbf{修复4}:对破坏性命令增加保护\\
\hspace{1em}• Compact 后一段时间内禁止 git checkout/reset/restore\\
\hspace{1em}• 或要求 agent 在执行前必须确认};

\end{tikzpicture}
\end{center}

%==============================================================================
\section{结论}
%==============================================================================

\begin{conclusionbox}
\textbf{问题性质}:

这是 Codex CLI 的一个\textbf{已知的、普遍存在的设计缺陷},在官方 GitHub 上有大量用户报告。

\vspace{1em}
\textbf{根本原因(第一性原理)}:

Compact 机制在压缩对话历史时,只保留简短的文字摘要,\textbf{丢失了关于 git 工作区状态的关键信息}。Compact 后的 agent 看到工作区有修改,但不知道这些是自己之前的工作,于是\textbf{主动执行操作进行"清理"}。

\vspace{1em}
\textbf{为什么 Execpolicy 不够}:

\begin{enumerate}
    \item Execpolicy \textbf{只能阻止 shell 命令},无法阻止 Edit/Write 等工具
    \item Agent 如果被禁止用 git checkout,可能转而用其他方式"修复"
    \item \textbf{治标不治本}:只是堵住了部分回滚手段,没有解决误判问题
    \item 真正的问题是 \textbf{Compact prompt 没有要求记录 git 状态}
\end{enumerate}

\vspace{1em}
\textbf{触发条件}:

\textbf{tmux 多窗口环境}下多 agent 并行工作会加速 token 消耗,更频繁触发 compact;共享仓库使得状态更复杂,agent 更容易误判。

\vspace{1em}
\textbf{当前可行方案}:

\begin{enumerate}
    \item \textbf{定时自动 commit \& push}(最有效的临时方案)
    \item 配置 execpolicy 禁止 git checkout/reset/restore 命令(部分有效)
    \item 增大 token limit 减少 compact 频率
    \item 在 AGENTS.md 中添加保护指令(效果有限,compact 后会被忽略)
\end{enumerate}

\vspace{1em}
\textbf{真正需要修复的}:

\textbf{Compact 机制本身}——需要在 compact prompt 中增加要求,让 agent 记录:
\begin{itemize}
    \item 当前 git 工作区状态和修改文件列表
    \item 明确声明"这些修改是之前工作的成果,不要撤销"
    \item 当前 git 分支和最近 commit hash
\end{itemize}
\end{conclusionbox}

\vspace{1cm}
\begin{center}
\rule{0.8\textwidth}{0.4pt}

\textit{本报告基于 codex-rs 源码分析、OpenAI 官方 GitHub Issues 调研与第一性原理分析}

\textit{版本 4.0 - \today}

\vspace{0.5em}
\small
相关 Issues:
\href{https://github.com/openai/codex/issues/5957}{\#5957},
\href{https://github.com/openai/codex/issues/1736}{\#1736},
\href{https://github.com/openai/codex/issues/6022}{\#6022},
\href{https://github.com/openai/codex/issues/4969}{\#4969},
\href{https://github.com/openai/codex/issues/2927}{\#2927},
\href{https://github.com/openai/codex/issues/4106}{\#4106}

\vspace{0.5em}
\small
关键源码文件:
\texttt{core/src/compact.rs},
\texttt{core/src/exec\_policy.rs},
\texttt{core/templates/compact/prompt.md}
\end{center}

\end{document}
