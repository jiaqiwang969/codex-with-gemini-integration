\documentclass[12pt]{ctexart}
\usepackage[a4paper,margin=2.5cm]{geometry}
\usepackage{hyperref}
\usepackage{listings}
\usepackage{xcolor}
\usepackage{enumitem}
\usepackage{longtable}

% 为 listings 定义 Rust 语言高亮,避免 language=Rust 报错
\lstdefinelanguage{Rust}{
  keywords={typeof, new, true, false, catch, function, return, null, switch, var,
    if, in, while, do, else, case, break, let, mut, match, use, mod, pub, struct,
    impl, enum, fn, async, await, move, trait, where, type, dyn, crate, self, Self},
  keywordstyle=\bfseries\color{blue!70!black},
  sensitive=true,
  comment=[l]{//},
  morecomment=[s]{/*}{*/},
  commentstyle=\itshape\color{green!40!black},
  stringstyle=\color{red!60!black},
  morestring=[b]",
  morestring=[b]'
}

\lstdefinestyle{code}{
  basicstyle=\ttfamily\small,
  breaklines=true,
  tabsize=2,
  columns=fullflexible,
  showstringspaces=false,
  keywordstyle=\bfseries\color{blue!70!black},
  commentstyle=\itshape\color{green!40!black},
  stringstyle=\color{red!60!black}
}
\lstset{style=code}

\title{Codex CLI 中 Gemini 3 Pro 接入技术报告}
\author{基于本地分支的集成实践}
\date{\today}

\begin{document}
\maketitle

\begin{abstract}
本文系统梳理了在 Codex CLI 中接入 Gemini 3 Pro Preview 的技术路线。
目标是让后续开发者能够看懂、照做、改进,而不是停留在模糊的术语层面。
报告重点围绕三个关键点展开:

\begin{itemize}
  \item 如何在 Codex 的通用架构下,把 Gemini 3 Pro(低思考版和高思考版)接入进来;
  \item 工具调用循环(function call loop)在底层是怎样运转的;
  \item thoughtSignature 是如何在整个调用链路中被接收、保存、回放的。
\end{itemize}

文中所有说明都以当前仓库的 Rust 代码为依据,尽量避免空洞的概念描述。
\end{abstract}

\tableofcontents

\section{当前集成的模型与目标}

\subsection{使用的 Gemini 模型}

当前 Codex 集成的是 Gemini 3 Pro Preview 模型族,具体包括:

\begin{itemize}
  \item \texttt{gemini-3-pro-preview}:基础版(低思考,默认不输出 thoughts);
  \item \texttt{gemini-3-pro-preview-codex}:为 Codex 行为微调 Prompt 的低思考变体;
  \item \texttt{gemini-3-pro-preview-thinking}:高思考版本,支持输出 thoughts;
  \item \texttt{gemini-3-pro-preview-thinking-codex}:高思考 + Codex 优化 Prompt。
\end{itemize}

也就是说,Codex 侧实际支持的是“Gemini 3 Pro Preview 3.0”的两个思考档位:

\begin{itemize}
  \item \textbf{Low thinking}:\texttt{gemini-3-pro-preview(-codex)};
  \item \textbf{High thinking}:\texttt{gemini-3-pro-preview-thinking(-codex)}。
\end{itemize}

在请求的 \verb|generationConfig| 中,我们用 \verb|thinkingConfig| 来区分
低思考和高思考模型:

\begin{itemize}
  \item 非 thinking 模型:\verb|thinkingLevel = "low"|,不主动请求 thoughts;
  \item thinking 模型:\verb|thinkingLevel = "high"|,且 \verb|includeThoughts = true|。
\end{itemize}

为了保证“低思考”版本在工具循环中同样能坚持“寻根究底”,我们对两类模型
\textbf{统一设置}了较大的思考预算:

\begin{itemize}
  \item \verb|thinkingBudget = 32768|(低思考、高思考版本完全一致)。
\end{itemize}

即使从官方文档角度看,Gemini 3 Pro 更推荐用 \verb|thinkingLevel| 控制思考量,
但 Codex 仍显式设置一个较高的 \verb|thinkingBudget|,目的是在代理或旧实现
仍检查此字段时,给出足够的预算,不让长链路工具调用过早被截断。

\subsection{集成目标}

从第一性原理出发,本次 Gemini 接入的目标可以拆成三点:

\begin{enumerate}
  \item \textbf{调用通路打通}:
  在不破坏 Codex 现有 OpenAI 流程的前提下,为 Gemini 单独实现
  HTTP 调用、SSE 解析、错误处理等逻辑,并与 OpenAI 提供方完全隔离。
  \item \textbf{工具调用行为对齐 GPT-5.1}:
  包括:敢用工具、连用工具、长时间用工具,而不是查两下就开始“编故事”。
  \item \textbf{上下文与 UI 行为一致}:
  TUI 中的 \verb|context left|、“运行中 / 就绪” 状态,对 Gemini 和 GPT-5.1
  都要准确,不因为换了模型就乱。
\end{enumerate}

\section{整体架构:从 Provider 到 TUI}

\subsection{ModelProviderInfo:把 Gemini 当成一个独立的 Provider}

Codex 使用 \verb|ModelProviderInfo| 描述每一个“模型提供方”,定义在
\verb|codex-rs/core/src/model_provider_info.rs|。关键字段包括:

\begin{itemize}
  \item \verb|name|:用于界面显示的名称;
  \item \verb|base_url|:HTTP API 的基础 URL;
  \item \verb|wire_api|:使用的协议类型(Responses / Chat / Gemini);
  \item \verb|auth_json_key| / \verb|env_http_headers|:
  如何从环境变量或 \verb|auth.json| 读取 key,并注入到请求头。
\end{itemize}

Gemini 的默认 provider 在 \verb|built_in_model_providers()| 中定义,大致如下(简化版):

\begin{lstlisting}[language=Rust]
("gemini",
 P {
     name: "Gemini (Preview)".into(),
     base_url: std::env::var("GEMINI_BASE_URL")
         .ok()
         .filter(|v| !v.trim().is_empty())
         .or_else(|| Some("https://api.ppchat.vip/v1beta".into())),
     auth_json_key: Some("GEMINI_API_KEY".to_string()),
     wire_api: WireApi::Gemini,
     env_http_headers: Some(
         [
             ("X-Goog-Api-Key".to_string(), "GEMINI_API_KEY".to_string()),
             ("Cookie".to_string(), "GEMINI_COOKIE".to_string()),
         ]
         .into_iter()
         .collect(),
     ),
     requires_openai_auth: true,
 })
\end{lstlisting}

这里有几点需要特别说明:

\begin{itemize}
  \item \textbf{OpenAI 与 Gemini 完全隔离}:
  OpenAI 使用 Responses API(\verb|/v1/responses|),Gemini 使用
  自己的 \verb|:generateContent| / \verb|:streamGenerateContent| 协议,
  两者通过 \verb|wire_api| 明确区分,走的是不同的 base\_url 和认证逻辑。
  \item \textbf{Base URL 可切换代理}:
  \verb|GEMINI_BASE_URL| 可以指向官方 GCP,也可以指向
  \verb|https://api.vectorengine.ai/v1beta| 等代理。
  Codex 不关心具体是哪家服务,只要遵守 Gemini 3 Pro 的协议即可。
  \item \textbf{Key 读取顺序}:
  优先从环境变量 \verb|GEMINI_API_KEY| 读取;如果缺失,再从
  \verb|auth.json| 中读取同名 key。对于部分代理,也支持从
  \verb|OPENAI_API_KEY| 复用共享 key。
\end{itemize}

\subsection{ModelClient 与 stream\_gemini:专门走 Gemini 协议的通路}

在 \verb|codex-rs/core/src/client.rs| 中,\verb|ModelClient| 为不同 provider
实现了不同的调用通路:

\begin{itemize}
  \item OpenAI Responses:\verb|stream_responses_api(...)|;
  \item Chat Completions:\verb|stream_chat_completions_api(...)|;
  \item Gemini:\verb|stream_gemini(...)|。
\end{itemize}

对 Gemini,调用的是 \verb|models/{api_model}:streamGenerateContent?alt=sse|,
使用 SSE 流式输出。整体流程是:

\begin{enumerate}[label=\arabic*)]
  \item 根据当前模型名(例如 \verb|gemini-3-pro-preview-codex|)去掉 \verb|-codex| 后缀,得到真正的 API 模型名 \verb|gemini-3-pro-preview|。
  \item 使用 provider 的 \verb|base_url| 构造 SSE URL。
  \item 将 Codex 内部的 \verb|Prompt| 转为 Gemini 的 \verb|contents|:
  既包括 system 指令,也包括历史消息(用户消息、模型回复、工具调用和工具结果)。
  \item 使用 \verb|build_gemini_tools(...)| 把 Codex 的工具定义映射为
  Gemini 的 \verb|functionDeclarations|。
  \item 生成 \verb|generationConfig|,配置温度、思考等级和思考预算等:
  \begin{itemize}
    \item \verb|temperature = 0.8|(比官方默认 1.0 略低,鼓励更稳定但仍有探索的推理);
    \item \verb|topK = 64|、\verb|topP = 0.95|;
    \item \verb|thinkingConfig| 如前所述(low/high + 统一的 \verb|32768| 预算)。
  \end{itemize}
  \item 从环境变量或 \verb|auth.json| 中读取 Gemini key,放到
  \verb|X-Goog-Api-Key| 请求头。
  \item 发送请求,并将返回的 SSE 字节流交给
  \verb|process_gemini_sse(...)| 解析。
\end{enumerate}

这一层的本质工作是:“把 Codex 的内部世界翻译成 Gemini API 要求的 JSON”,
并在这一层处理好错误和重试,让上层逻辑保持干净。

\subsection{常见错误与重试策略}

在 \verb|stream_gemini| 中,Codex 对网络错误、429 和 5xx 做了统一处理:

\begin{itemize}
  \item 至多 3 次重试,指数退避(5000ms 起步,含随机抖动,上限 30000ms);
  \item 对纯网络错误(连接中断、超时)同样重试;
  \item 若连续失败,则向上抛出 \verb|ResponseStreamFailed|,由上层展示错误。
\end{itemize}

如果上游代理返回的是中文错误如:

\begin{quote}
\verb|429 Too Many Requests: 该令牌状态不可用|
\end{quote}

这类错误通常表示“当前 token 不可用或被限流”,而不是 Codex 侧的协议错误。
可以直接用手工 \verb|curl| 复现(例如使用 \verb|:generateContent|),确认
代理本身是否健康。

此外,还有一类特殊的 400/429 错误:如果上游认为某次工具调用缺少
thoughtSignature,会返回“missing a \verb|thought_signature|”之类的信息。
对于这种情况,Codex 会兜底生成一条普通的 assistant 消息,把错误解释展示给用户,
而不是直接让整次对话失败(后文会详细说明 thoughtSignature 的处理)。

\section{工具调用循环:模型、工具和 Codex 之间的闭环}

\subsection{一个完整的 function call loop 是怎样运转的?}

从 Codex 视角,一次“Gemini 带工具”的调用循环可以拆成四步:

\begin{enumerate}[label=\arabic*)]
  \item \textbf{发起请求}:带上完整上下文(包括以往的 functionCall 和 functionResponse)。
  \item \textbf{解析 SSE 输出}:从模型输出中抓出文本增量和 functionCall。
  \item \textbf{执行工具}:遇到 functionCall 就调用本地工具(shell、MCP 等),得到结果。
  \item \textbf{回放工具结果}:把 functionCall 和 functionResponse 作为新的一对消息,加到历史里,再发给模型。
\end{enumerate}

Codex 的核心循环在 \verb|codex-rs/core/src/codex.rs| 的
\verb|try_run_turn(...)| 中,大致逻辑是:

\begin{itemize}
  \item 使用当前 Prompt 调用 \verb|client.stream(...)|(Gemini 时即为 \verb|stream_gemini|),得到一个 \verb|ResponseEvent| 流;
  \item 监听这个流:
  \begin{itemize}
    \item 文本增量 \verb|OutputTextDelta| → 直接显示;
    \item \verb|OutputItemDone(item)|:
    如果能解析出工具调用(\verb|ToolRouter::build_tool_call| 返回 Some),说明模型发了 functionCall,这时:
    \begin{enumerate}[label=\arabic*)]
      \item 做 loop 检测(避免死循环,后文详述);
      \item 把 functionCall 交给 \verb|ToolCallRuntime| 执行;
      \item 把执行结果打包成 \verb|FunctionCallOutput|,记录到上下文;
    \end{enumerate}
    如果解析不出工具调用,就是普通消息,按文本处理。
  \end{itemize}
  \item 执行完工具后,Codex 会基于更新后的历史再次调用模型,如此往复,直到没有新的 functionCall 为止。
\end{itemize}

这就是你看到的“Gemini 连续 functionCall”的真实结构:每一轮都是
“模型 → 工具调用 → 工具执行 → 工具结果回放 → 再次调用模型”的闭环。

\subsection{SSE 解析:从 JSON 分块到 ResponseEvent}

解析逻辑在 \verb|process_gemini_sse(...)| 中。核心思路:

\begin{lstlisting}[language=Rust]
loop {
    let sse = next_sse_chunk()?;
    if sse.data.trim().is_empty() { continue; }

    let chunk: GeminiResponse = serde_json::from_str(&sse.data)?;

    // 更新 response_id 和 usage_metadata
    if let Some(id) = chunk.response_id { last_response_id = id; }
    if let Some(usage) = chunk.usage_metadata { last_token_usage = Some(usage.into()); }

    // 处理 candidates[*].content.parts[*]
    if let Some(candidates) = chunk.candidates {
        for candidate in candidates {
            if let Some(content) = candidate.content
                && let Some(parts) = content.parts {
                for part in parts {
                    // 1) 记录 thoughtSignature
                    if let Some(sig) = &part.thought_signature {
                        last_thought_signature = Some(sig.clone());
                    }

                    // 2) 文本内容 → OutputTextDelta
                    if let Some(text) = part.text && !text.is_empty() {
                        // 第一次文本先发 OutputItemAdded(Message)
                        // 然后每块发 OutputTextDelta
                    }

                    // 3) functionCall → 记下 (name, args, thoughtSignature)
                    if let Some(call) = part.function_call {
                        ...
                        function_call = Some((call.name, args, part.thought_signature
                            .or(last_thought_signature.clone())));
                    }
                }
            }
        }
    }
}
\end{lstlisting}

流结束时,如果存在 \verb|function_call|,就发出一条
\verb|ResponseItem::FunctionCall|,否则发一条普通的文本消息。

\section{thoughtSignature 的完整链路}

\subsection{thoughtSignature 是什么?我们对它做了什么?}

在 Gemini 3 Pro 中,\verb|thoughtSignature| 是服务端用来约束多轮
function calling 的“思考步骤标识”。文档给出的关键信息是:

\begin{itemize}
  \item 每一个 functionCall / functionResponse 所在的“思考步”都会带一个签名;
  \item 服务端在多轮调用中会检查这些签名是否一致、是否缺失;
  \item 签名本身对客户端是“不透明的字符串”,客户端只需要正确传递。
\end{itemize}

Codex 的策略很简单:\textbf{不解释含义,只做正确搬运和兜底}:

\begin{itemize}
  \item 在解析 SSE 时,把所有出现的 thoughtSignature 记录下来;
  \item 在回放 functionCall 和 functionResponse 时原样传回;
  \item 在历史被裁剪或重建导致缺失时,按官方建议填入一个特殊值,让服务端验证通过。
\end{itemize}

\subsection{部分一:从 SSE 中找到 thoughtSignature}

如前所述,在 \verb|process_gemini_sse| 中,任何包含
\verb|part.thought_signature| 的块,都会更新 \verb|last_thought_signature|。
当一个 part 同时包含 \verb|functionCall| 时,会优先使用该 part 上的签名;
若为空,则使用 \verb|last_thought_signature| 兜底。

最终发出的 \verb|ResponseItem::FunctionCall| 结构是:

\begin{lstlisting}[language=Rust]
ResponseItem::FunctionCall {
    id: None,
    name,
    arguments,
    call_id: "gemini-function-call".to_string(),
    thought_signature, // Option<String>
}
\end{lstlisting}

这一步的目的就是:把模型 SSE 中散落的 thoughtSignature 信号,整理成“这一轮函数调用对应的签名”。

\subsection{部分二:把 functionCall 和 functionResponse 串起来}

下一轮请求时,我们会把历史里的 \verb|FunctionCall| 和
\verb|FunctionCallOutput| 重新编码为 Gemini 所需的 \verb|contents|。
代码在 \verb|build_gemini_contents(...)| 中。

\paragraph{FunctionCall 部分}

\begin{lstlisting}[language=Rust]
ResponseItem::FunctionCall { name, arguments, thought_signature, .. } => {
    last_function_call_name = Some(name.clone());
    last_function_call_thought_signature = thought_signature.clone();
    let args: serde_json::Value = serde_json::from_str(arguments)
        .unwrap_or(serde_json::Value::Object(Default::default()));
    let part_thought_signature = thought_signature.clone();
    contents.push(GeminiContentRequest {
        role: Some("model".to_string()),
        parts: vec![GeminiPartRequest {
            text: None,
            inline_data: None,
            function_call: Some(GeminiFunctionCallPart { name: name.clone(), args }),
            function_response: None,
            thought_signature: part_thought_signature.clone(),
            compat_thought_signature: part_thought_signature,
        }],
    });
}
\end{lstlisting}

要点:

\begin{itemize}
  \item role 设为 \verb|"model"|,表示这是模型上一次发出的函数调用;
  \item \verb|thoughtSignature| 原样放在这个 part 上,同时在
  \verb|compat_thought_signature| 上再打一份,以兼容某些只认 snake\_case 的代理;
  \item \verb|last_function_call_name| 和
  \verb|last_function_call_thought_signature| 被保存下来,供下一条 functionResponse 使用。
\end{itemize}

\paragraph{FunctionCallOutput 部分}

\begin{lstlisting}[language=Rust]
ResponseItem::FunctionCallOutput { output, .. } => {
    let function_name = last_function_call_name
        .take()
        .unwrap_or_else(|| "unknown_function".to_string());
    let thought_signature = last_function_call_thought_signature.take();

    let response_value = serde_json::json!({
        "output": output.content.clone(),
        "success": output.success.unwrap_or(true)
    });
    let part_thought_signature = thought_signature.clone();

    contents.push(GeminiContentRequest {
        role: Some("function".to_string()),
        parts: vec![GeminiPartRequest {
            text: None,
            inline_data: None,
            function_call: None,
            function_response: Some(GeminiFunctionResponsePart {
                name: function_name,
                response: response_value,
            }),
            thought_signature: part_thought_signature.clone(),
            compat_thought_signature: part_thought_signature,
        }],
    });
}
\end{lstlisting}

也就是说:\textbf{同一个工具调用的 functionCall 和 functionResponse,共用同一个 thoughtSignature}。
这正是 Gemini 3 Pro 文档所要求的行为。

\subsection{部分三:缺失签名时的兜底}

在真实环境中,有时历史会因为裁剪、合并或代理行为而丢失签名。
为避免这类情况直接导致 400/429,我们增加了
\verb|ensure_active_loop_has_thought_signatures(...)|:

\begin{itemize}
  \item 找到“当前活跃 loop”的起点:最后一个带文本内容的 user 消息;
  \item 从这一点之后的所有 role=\verb|"model"| 的内容中,找到第一个
  \verb|functionCall| part;
  \item 如果它缺少 \verb|thoughtSignature|,就填入一个约定的 dummy 值
  \verb|"skip_thought_signature_validator"|。
\end{itemize}

这样可以保证:即使我们对历史做了技术性处理,发送到 Gemini 的
“当前回合函数调用”始终满足服务端验证器的要求。

\section{工具调用 loop 检测与防止死循环}

\subsection{为什么只对 Gemini 做 loop 检测?}

在 \verb|codex-rs/core/src/codex.rs| 的
\verb|try_run_turn(...)| 中,我们引入了一个轻量级 loop 检测逻辑:

\begin{lstlisting}[language=Rust]
let model_family = turn_context.client.get_model_family();
let detect_tool_call_loops = model_family.family == "gemini";
const TOOL_CALL_LOOP_THRESHOLD: i32 = 100;
let mut last_tool_call_key: Option<String> = None;
let mut tool_call_repetition_count: i32 = 0;
\end{lstlisting}

只对 family 为 \verb|"gemini"| 的模型启用,原因有两点:

\begin{enumerate}
  \item 在早期测试中,确实观察到 Gemini 可能在“意外卡死”时不断用完全相同的参数重复发同一个工具调用;
  \item 我们不希望对 GPT-5.1 等模型引入额外限制。
\end{enumerate}

\subsection{具体检测规则}

当 \verb|ToolRouter::build_tool_call(...)| 返回一个工具调用时,我们构造一个稳定的 key:

\begin{lstlisting}[language=Rust]
let payload_preview = call.payload.log_payload().into_owned();
let key = format!("{}:{}", call.tool_name, payload_preview);
\end{lstlisting}

然后:

\begin{itemize}
  \item 如果 key 与上一次相同,则 \verb|tool_call_repetition_count += 1|;
  \item 否则重置为 1,并更新 \verb|last_tool_call_key|;
  \item 当计数达到 \verb|TOOL_CALL_LOOP_THRESHOLD = 100| 时,认为进入死循环。
\end{itemize}

此时,我们不会再次执行工具,而是构造一个失败的函数调用输出:

\begin{lstlisting}[language=Rust]
let loop_message = format!(
    "Loop detected: tool '{}' was called repeatedly \
with identical arguments. Stopping further tool calls for this turn to avoid \
an infinite loop. If you intended to keep running this command, please explain \
why and rephrase your request.",
    call.tool_name,
);

let response = ResponseInputItem::FunctionCallOutput {
    call_id: call.call_id.clone(),
    output: FunctionCallOutputPayload {
        content: loop_message,
        success: Some(false),
        ..Default::default()
    },
};
\end{lstlisting}

这条消息被当作正常的 \verb|FunctionCallOutput| 返回给模型和 UI,使用户看到清晰的解释,同时终止当前 turn 后续的工具调用。

注意,这里只对“完全相同的工具名 + 参数”进行统计,所以正常“多轮 bash 搜索”不会触发此保护。

\section{TUI 集成:上下文窗口与运行状态}

\subsection{上下文窗口与 context left 指示器}

Gemini 3 Pro Preview 的上下文窗口为约 1M tokens。
为了让 TUI 底部的 “xx\% context left” 正确反映这一点,我们在
\verb|codex-rs/core/src/openai_model_info.rs| 中为
\verb|slug.starts_with("gemini-")| 的模型配置:

\begin{lstlisting}[language=Rust]
pub(crate) const CONTEXT_WINDOW_1M: i64 = 1_000_000;

_ if slug.starts_with("gemini-") => Some(ModelInfo::new(CONTEXT_WINDOW_1M)),
\end{lstlisting}

同时,在 \verb|Session::update_token_usage_info(...)| 中,当
Gemini 流式响应缺少 usage 元数据时,我们会调用
\verb|recompute_token_usage(...)| 基于历史估算 token 使用量,
不会让 context left 永远停在 100\%。

\subsection{“运行中” / “就绪” 状态的修正}

最初 TUI 侧边栏的状态字符串只依赖底部状态指示器是否可见和 exec 活动。
在长时间 Gemini 任务下,这会出现“流式内容结束后就被误判为就绪”的问题。

为此我们在 \verb|ChatWidget::sidebar_status(...)| 中改为:

\begin{lstlisting}[language=Rust]
pub(crate) fn sidebar_status(&self) -> String {
    // 只要底部 Pane 认为有任务在运行,就视为「运行中」,不依赖状态指示器是否可见。
    if self.bottom_pane.is_task_running() {
        return "运行中".to_string();
    }

    // 其次看 execCell 和 running_commands
    ...

    "就绪".to_string()
}
\end{lstlisting}

这里的 \verb|is_task_running| 是由核心任务事件驱动的:只有当整个 turn 确认完成,才会被置为 false。
这样,即使为了 UI 效果临时隐藏底部状态指示器,侧边栏仍然可以正确反映 Gemini 任务的真实状态。

\section{演进路径概览与对后续接入的启示}

从本地起点(例如 \verb|914f31b1c7db...|)到当前最新提交,围绕 Gemini 的工作大致经历了以下阶段:

\begin{enumerate}[label=\arabic*)]
  \item 引入 Gemini provider 与基本模型预设;
  \item 实现 \verb|stream_gemini|,打通 SSE 流式通路;
  \item 把 \verb|thinkingLevel| / \verb|thinkingConfig| 接到 generationConfig 上;
  \item 统一设置较大的 \verb|thinkingBudget = 32768|,让 low/high 版本都能支撑长链路工具调用;
  \item 实现 thoughtSignature 的完整传递:解析、存储、回放,并在缺失时兜底;
  \item 增加针对 Gemini 的工具调用 loop 检测,阈值从 5 提升到 100;
  \item 修复 JSON Schema 不兼容 \verb|additionalProperties| 的问题;
  \item 在 TUI 中修正 context left 与 “运行中 / 就绪” 状态,使之与 Gemini 的真实行为一致。
\end{enumerate}

这些实践为未来接入其他 API 提供了两点重要经验:

\begin{itemize}
  \item \textbf{协议适配要在边缘做透}:
  SSE、JSON Schema、错误格式等差异,最好集中在 provider 和 client 层消化,
  上层统一用 Codex 自己的模型表示。
  \item \textbf{工具循环和思考链要端到端设计}:
  不仅要会发送 functionCall,还要正确回放 functionResponse 和 thoughtSignature,
  并通过 Prompt、预算和 loop 检测配合起来,才能得到稳定的“长时间寻根究底”行为。
\end{itemize}

\end{document}
